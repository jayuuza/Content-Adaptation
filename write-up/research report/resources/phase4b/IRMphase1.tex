\documentclass[11pt]{article}
\usepackage{witsa4}
\usepackage{times}

\usepackage{url}
\usepackage{natbib} % Force natbib.sty to put citation labels in the reference list
\makeatletter
\renewcommand\NAT@biblabel[1]{\def\citeauthoryear##1##2{##1 ##2}[#1]\hfill}
\renewcommand\NAT@bibsetup[1]{%
  \setlength{\itemsep}{\bibsep}\setlength{\parsep}{\z@}}
\def\@lbibitem[#1]#2{%
  \if\relax\@extra@b@citeb\relax\else
    \@ifundefined{br@#2\@extra@b@citeb}{}{%
     \@namedef{br@#2}{\@nameuse{br@#2\@extra@b@citeb}}}\fi
   \@ifundefined{b@#2\@extra@b@citeb}{\def\NAT@num{}}{\NAT@parse{#2}}%
   \item[\hfil\hyper@natanchorstart{#2\@extra@b@citeb}\@biblabel{#1}%
    \hyper@natanchorend]%
    \NAT@ifcmd#1(@)(@)\@nil{#2}}
\makeatother


\bibliographystyle{named-wits}
\bibpunct{[}{]}{;}{a}{}{,}  

\title{How Students Use the Internet:\\
A comparison of two research papers}

\author{Jonathan Wasson 463117}

\begin{document}


\maketitle

\clearpage

\tableofcontents
\clearpage


\section{Introduction}

The internet is quickly becoming ingrained in the way humans interact with the world. It's every increasing popularity and widespread use, which affects more aspects of business, academia and even social interactions. Two studies were conducted to look into these effects. In particular, the way students use the internet to conduct research. Special emphasis was placed on whether or not students were capable of discerning valid data from that which was unverifiable.
\\

The purpose of this document is to take a close look at both papers. Looking at the methodology used, the focus, the purpose of each paper etc. in an attempt to compare and contrast them. Giving input where necessary and scrutinizing each paper's overall effectiveness. 
\\

The two papers will be referred to as paper 1 and 2 in this Comparison document.
\begin{itemize}
\item Paper 1: "Of course its true; I saw it on the internet! By Leah Graham and Panagiotis Takis Metaxas"

\item Paper 2: "College student Web use, perceptions of information credibility, and verification behavior by Miriam J. Metzger, Andrew J. Flanagin, Lara Zwarun"
\end{itemize}


\section{General Overview}

It's important to not that these papers were released in 2003 when much of what modern students take for granted was not popular. Information on the internet was harder to obtain and much harder to verify. Websites such as Wikipedia had not gained much popularity and as Wikipedia has gotten older, it has begun hiring actual specialists to check articles and to write articles. Which wasn't the case when these papers were written. \citep{voss2005measuring}
\\

Both papers focused on students abilities to discern whether or not information found on the internet was valid (i.e. looking at the accuracy of the information). However the basic validity and current day relevance of this question can be brought into question. There are many reasons why this question doesn't even apply nowadays. \citep{van2011internet} With research becoming harder and harder to accomplish without using the internet due to a decrease in popularity of print media. An increase in internet popularity has saturated the internet with data but many sources suggest this is no cause for concern regarding the accuracy of said data. A concept called Swarm Intelligence suggests that the more people changing sources such as Wikipedia, the more accurate the source becomes. Basically saying that the collective intelligence will tend towards an accurate or correct concept. \citep{kennedy2001swarm} So, with the massive amount on the internet today and the apparent increase in skilfulness of it's users, it is safe to say these papers aren't as relevant as they once were.
\clearpage

\section{Target Audience}

\subsection{Paper 1}

The first paper: "Of course its true; I saw it on the internet! By Leah Graham and Panagiotis Takis Metaxas" \citep{graham2003course} appears to be written for publication in an easy to read manner, not for use by professionals. The paper was also quite concise without any lengthy, detailed explanations. Rather as an easy to read paper without much focus on research methodology but with focus given to the results of said research. The style and writing of the paper is much easier to read than the second paper. All this gives the impression that it's target audience isn't professional but rather average/everyday readers.
\\

The layout of the document is much more visually appealing than the 2nd paper. A reference list with clearly labelled and listed references is a good example of this. The addition of a 2nd column of text on one page makes it easier to keep track of position when reading. Correct use of headings and paragraphs made the logical flow of ideas in the paper easy to follow. The display of data was very easy to read as well with only one complaint. Due to the double columns of text, tables of data were often  in strange places with little to connect them to where they were referenced in the text. Besides this, the paper was easy to read.
\\

\subsection{Paper 2}

The second paper: "College student Web use, perceptions of information credibility, and verification behavior by Miriam J. Metzger, Andrew J. Flanagin, Lara Zwarun" \citep{metzger2003college} is aimed at a level suitable to researchers and professionals. The style would be considered incompatible to people without any experience in the topics and methods used in the paper. The notation used is also unfamiliar to most people not experienced with computer science or perhaps even statistics. 
\\

The logical layout of this document is also incredibly lacking in foresight. Section headings and subsections are not placed in a way that is easy to read, sometimes without a logical flow from idea to idea. The pages seem to run into it each other making it very hard to keep your place. The authors could have simply added paragraphs to separate ideas from one another. This lack of paragraphs is not only visually unappealing to the reader but makes it hard to keep track of ones place.
\\

The tables included in the paper along was represented in way that made it hard to read. Data such as the ample amount of statistics was provided in the same manner as the above tables, overwhelming the senses.
Even though the paper is aimed at a professional audience, the style and formatting is severely lacking. The reference list is a perfect example of this, with a very large amount of references listed but with no spacing in numbering. People don't want to sift through a wall of text to find a single reference.

\clearpage
\section{Differences in Research methodology}

\subsection{Paper 1}

The research in this paper was done by taking college students and giving them questions to answer where the way they gathered data on the questions was analysed. The results of this exercise showed that students preferred to use the internet as a means of gathering data. With indication that most students used search engines to achieve this. While all this is very interesting, the paper did not really say how the research was done. Nor did it provide any indication as to how many people took part in the study or any other factors. Given that the target audience would not be interested in these details, it is still important to note them. 
\\

Special cases were made in the experiment to see how students reacted to different types of misinformation. They did not however say how many students answered these questions. This is a highly important factor regarding these types of experiments. Since the amount of students participating in a study shows how accurate the study was. A small amount of sample data isn't very useful in a statistical analysis. These factors are important in these kinds of studies.  Due to this, it is easy to question the validity of the methods used in the research and brings into question the truth of the paper.
\\

While the paper did ask the students questions which are quite useful, it did not convey in any way the method used to ask these questions. The way in which a question is asked could drastically affect the studies outcome. Advertising seemed to be the main 'theme' of misinformation researched in the paper, with little emphasis given to other types. No research was done into more everyday types of misinformation.


\subsection{Paper 2}
The focus of paper 2 is more to the point with better emphasis on actual web usage as opposed to things such as advertising. It includes a very structured layout of the questions used along with an actual overview of the statistics involved. Unlike the first paper, the inclusion of these statistics gives more weight to the research. Where the first paper jumped around in explaining exactly what the students were using the internet for without actually getting students to use the internet anywhere in their research, paper 2 goes into detail on how students use  the internet.
\\

The methods used in the research were carefully laid out in the paper, again giving even more weight to the results of the paper. However the authors did not try summarise their results in an easy to read manner. Which made it seem messy detracting from the great research done. Unnecessary data was included as well bloating the paper further.
\\

Neither paper was able to show the validity of information. In fact there probably is no way to determine how accurate information on the internet is. Both papers seemed to ask the students whether the content was accurate or not without any method to indicate whether it is or not.






\clearpage
\section{General comments and criticisms}
\begin{itemize}
\item The second paper \citep{metzger2003college} contains an over abundance of information. Most of this information detracts from the paper by making it much harder to read. The writers should have rather tried to make this information easier to read, perhaps by tabulating it. 
\item  The authors might have been biased and perhaps misinformed regarding students. The author of the second paper often refers to students 'cavalier attitudes' \citep[page. 274]{metzger2003college} amongst other things.
\item Neither paper researched took into account content providers. Students are more likely to go to a website that is more reputable. They may have heard of such sites from friends and as a result reliable content providers will garner more hits than those that are not. While paper 1 did make mention of advertisers misinforming, this often does not crop up very often in ways that will affect the way they do research.
\item The two studies laid out in paper 2 were very well thought out with all the details pertaining to them given but they might have been too overcomplicated for the average person to read.
\item Both papers are out of date. Since the internet has come a long way in the past decade and the current set of college students grew up with the internet and its safe to say they have a better chance at sifting through information.
\item Paper 1's test subjects seemed to be aware they were part of research which may have affected the way they approached the problems.
\item Paper 1 seems to have a very confusing research method without any logical steps involved.
\item Paper 2 says that non-students find the web less valuable as a resource than students. Which isn't backed up by any data. Statements with no proof are rife in this paper.
\end{itemize}


\clearpage
\section{Conclusion} 

It could be said that both papers were written for a particular audience and as such were able to leave out certain things. 
\\

Both papers could have done with some serious improvements though. The one provided a great analysis of the studies done but the work was presented in a disjointed overly exhaustive way that made it a chore to read. The other was structured very neatly with little to complain about regarding structure but the analysis was oversimplified with the content pitched in a way lacking of meaning.
\\

These papers could have learned from one another. A combination of these papers would have yielded a neat, concise and informative analysis of the question. By taking paper one's format and styling with a few modifications and taking the research and methodology of paper two and combining them. With a few tweaks here and there

\clearpage
\bibliography{references}
\clearpage
\section{Updated time plan}
Completing Phase 1 of the hand-ins has given me some insight into the in how to better prepare for future hand-ins. The main problem that I recognised was that my schedule was often not met due to other duties. Completing more tasks on one day would be better than doing them over several days. The time allocated for each task wasn't very accurate either. I need to give more thought into how much time is allocated for each task.
\\


\begin{center}
	\begin{tabular}{ | p{5cm} | p{2cm} | p{3cm} | p{3cm} | p{3cm} |}
		\hline
		Task & Time       & Start   & End & Progress \\ \hline 
		Read first paper  & 2 hours & 2/11/2014 & 2/11/2014 & Complete  \\ \hline 
		Read second paper & 30 minutes & 2/11/2014 & 2/11/2014 & Complete  \\ \hline 
		Write down visible differences & 1 hour & 2/11/2014 & 2/11/2014 & Complete  \\ \hline 
		Reread and list comparisons & 2 hours & 2/11/2014 & 2/11/2014 & Complete  \\ \hline 
		Write document& 4 hours & 2/14/2014 & 2/14/2014 & Complete  \\ \hline 
		Proofread & 1 hour & 2/15/2014 & 2/15/2014 & Complete  \\ \hline 
		Prepare final draft & 4 hours & 2/17/2014 & 2/17/2014 & Complete \\ \hline 
	\end{tabular}
\end{center}


\end{document}
